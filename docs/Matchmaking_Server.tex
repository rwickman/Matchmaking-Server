\documentclass[conference]{IEEEtran}
\IEEEoverridecommandlockouts
% The preceding line is only needed to identify funding in the first footnote. If that is unneeded, please comment it out.
\usepackage{cite}
\usepackage{amsmath,amssymb,amsfonts}
\usepackage{algorithmic}
\usepackage{graphicx}
\usepackage{textcomp}
\usepackage{xcolor}
\def\BibTeX{{\rm B\kern-.05em{\sc i\kern-.025em b}\kern-.08em
    T\kern-.1667em\lower.7ex\hbox{E}\kern-.125emX}}
\begin{document}

\title{Distributed Multiplayer Video Game}

\author{\IEEEauthorblockN{Ryan Wickman}
\IEEEauthorblockA{\textit{University of Memphis} \\
Memphis, USA \\
rwickman@memphis.edu}
}

\maketitle

\begin{abstract}

\end{abstract}

\section{Introduction}
In this section I will provide a brief overview of the project and its implementation details.

\subsection{Goal}
The goal of this project initially was to create a full peer-to-peer multiplayer video game.
However, due to time constraints, I switched the scope of it to mainly focus on the matchmaking Server.
Thus my goal was updated to establish a good baseline matchmaking Server that players of a Video Game could use to find and join a game session.
Although this was my goal, I still did work on other components as time allowed, details of which I will give in the next section and throughout this paper.

\subsection{Overview}
In this project, I worked on a few components of what make up a mutliplayer video game.
What this entailed was making a way for users to send information amongst one another once in the game, a matchmaking server users could use to connect to a game, and the game itself.
I decided to use a peer-to-peer architecture for the in game communication for users. 
While this may cause more latency than a dedicated, centralized server, it scales better and is more cost efficent for myself.
In this, one player is chosen to host the game and act like the server.
The rest of the players will communicate through the host user as if it was a dedicated server itself.
They will use UDP to communicate as the game is a real-time application and is thus time sensitive.
On the other hand, the matchmaking server will be centralized as there needs to be a single point for all users to connect and express interest in finding a game session to join.
The users will use TCP, as opposed to UDP, to connect to the matchmaking server as reliable data transfer is important for finding and joining a game.
The game was developed using the Unity game engine. 
It is has a main menu that users can use to find a game through the matchamking server or connect to a game directly by using an IP address of the host.
The actual gameplay is a first-person sword fighting game.
While I have a good working example for these components, this is far from the final product it will eventually be.
Thus throughout this paper I will provide incite to where I beleive the application could be improved or expanded on.

\section{Matchmaking Server}
The majority of the time I spend on this project was focused on the matchmaking server.
It ended up having a lot more moving components than I first anticipated.
I used C++ to code everything and CMake to manage the build procces, and I used Boost.Asio to provide asynchronous networking capability.
The server is made up of a few parts: matchmaking server interface, TCPConnection, and game queue.

\subsection{Mathmaking Server Interface}
This is the point in which the user will first connect to the server.
The server is listening on a specified port and accepts incoming requests when the arrive. 
It asynchronously accepts the requests then initiates a callback that handles setting up the TCP socket to the client.
This allows for multiple users to connect to the server at once. 
The socket is constructed by initializing a TCPConnection object that will handle the rest of the users communication to find a game.
While I did consider creating a new thread for each connection, I did not due to time constraints and not wanting to deal with the complexity of adding multithreading such as locking resources, race conditions, ect..
When I update this application in the future I will potentially add this feature.

\subsection{TCPConnection}
The bulk of my time that I spent on the matchmaking server was on the TCPConnection class.
This is due to many factors such as redesigns, updating call sequences, callbacks, and simply just the overall complexity of this component.
The job of this component pretains to handling the communication with the client that allows them to connect to a game.
It does so by sending packets of data that are detailed in particular format.
The data in the packets are in JSON data format, which uses key-value pairs. 
This has to be serialized before it is sent over the network and deserialized when it arrives to its destination.
This works especially well since the data exchange is between the Video Game which is programmed in C# and the server which is programmed in C++ and there are libraries in both that support the use of this data format.
The packets themselves were designed to be lightweight and easily extensible.
While there are a many different packet types I created throughout this project, they all inherit from the parent class Packet.
The Packet class has members for the header length, maximum body length, body length, a char array to store the data, packet type, functions to access these members, functions to decode and encode the header, and functions to encode and decode the body.
The functions to encode and decode the body are pure virtual as to enforce the children classes to provide their own implementation to encode and decode the body.
The reason for doing this is because the information contained in the body of a packet differs between packet types.
While both the header and the body of a packet are stored in the same char array, they are separate entities.
The header is what contains the amount of bytes the body of the packet holds.
This is used to send variable sized packets and so that the receiver of a packet knows exactly how many bytes to read.
The header, however, is always the same size.
It currently is only 8 bytes however this and the maximum body length can be changed depending on the requirements of an inheriting child type. \par
Now, after talking about the data format it uses for interoperability, I will begin explaining the actua job of the TCPConnection class.
After the client has established a TCP connection with the Matchmaking server, a TCPConnection object is instantiated.
Then the first thing the client will do is send a FindGamePacket.
This is one of the packet types that inherit from the Packet superclass.
All of the subclass types will have the naming scheme of "XXXPacket".
The FindGamePacket object contains additional members for a unique identifier for the client and the game type the user wants to join.
This may change in the future to include more information, but this suffices for now.
Once the server begins to recieve the data for the FindGamePacketin an asynchronous read operation, it will first only read 8 bytes for the header.
Then, if this is successful, it will decode the header and read the amoount of bytes specified for the body.
This is done by calling a function from the callback function that reads the header of the packet.
This function called "TCPConnection::do\textunderscore read\textunderscore find\textunderscore game\textunderscorebody" will initiate another asychronous read operation to read the body.
Then once it is done reading the body, it will fire a callback that will add the user to a queue for the specified game type.
The way the user is pushed on the queue is by first getting a shared pointer to the queue from a queue manager.
Next, it will create a User object.
Finally, the User object is pushed on the queue.

\subsection{Game Queue}

\subsection{User}
The User class was created so that an item could be added to the game queue that is representative of a unique user.
Additionally, it contains smart pointers that contain references for callback functions that are used to allow the user to join or host a game.
Originally I planned on having a reference to the TCPConnection object in the User class.
However, the problem to this is it would create a circular dependency with the GameQueue class.
This is because the GameQueue class includes the header for the User class declarations and the TCPConnection includes the header for the GameQueue class declarations.
So, instead the User has reference to the functions "TCPConnection::host\textunderscore game and "TCPConnection::host\textunderscore game" to prevent such a circular dependency, but still have the necessary functionality.
This also provides more control for the TCPConnection to decide and handle what exact functions gets called.
In the future, this may prove helpful if I require different users to behave differently when joining or hosting a game.
One such instance would be if a private game is being created where only select users are able to join.
This join callback could provide additional authentication statements to verify a user can join a game before making a failed attempt and the TCPConnection closing thinking the user found a game.




\section{Prepare Your Paper Before Styling}
Before you begin to format your paper, first write and save the content as a 
separate text file. Complete all content and organizational editing before 
formatting. Please note sections \ref{AA}--\ref{SCM} below for more information on 
proofreading, spelling and grammar.

Keep your text and graphic files separate until after the text has been 
formatted and styled. Do not number text heads---{\LaTeX} will do that 
for you.

\subsection{Abbreviations and Acronyms}\label{AA}
Define abbreviations and acronyms the first time they are used in the text, 
even after they have been defined in the abstract. Abbreviations such as 
IEEE, SI, MKS, CGS, ac, dc, and rms do not have to be defined. Do not use 
abbreviations in the title or heads unless they are unavoidable.

\subsection{Units}
\begin{itemize}
\item Use either SI (MKS) or CGS as primary units. (SI units are encouraged.) English units may be used as secondary units (in parentheses). An exception would be the use of English units as identifiers in trade, such as ``3.5-inch disk drive''.
\item Avoid combining SI and CGS units, such as current in amperes and magnetic field in oersteds. This often leads to confusion because equations do not balance dimensionally. If you must use mixed units, clearly state the units for each quantity that you use in an equation.
\item Do not mix complete spellings and abbreviations of units: ``Wb/m\textsuperscript{2}'' or ``webers per square meter'', not ``webers/m\textsuperscript{2}''. Spell out units when they appear in text: ``. . . a few henries'', not ``. . . a few H''.
\item Use a zero before decimal points: ``0.25'', not ``.25''. Use ``cm\textsuperscript{3}'', not ``cc''.)
\end{itemize}

\subsection{Equations}
Number equations consecutively. To make your 
equations more compact, you may use the solidus (~/~), the exp function, or 
appropriate exponents. Italicize Roman symbols for quantities and variables, 
but not Greek symbols. Use a long dash rather than a hyphen for a minus 
sign. Punctuate equations with commas or periods when they are part of a 
sentence, as in:
\begin{equation}
a+b=\gamma\label{eq}
\end{equation}

Be sure that the 
symbols in your equation have been defined before or immediately following 
the equation. Use ``\eqref{eq}'', not ``Eq.~\eqref{eq}'' or ``equation \eqref{eq}'', except at 
the beginning of a sentence: ``Equation \eqref{eq} is . . .''

\subsection{\LaTeX-Specific Advice}

Please use ``soft'' (e.g., \verb|\eqref{Eq}|) cross references instead
of ``hard'' references (e.g., \verb|(1)|). That will make it possible
to combine sections, add equations, or change the order of figures or
citations without having to go through the file line by line.

Please don't use the \verb|{eqnarray}| equation environment. Use
\verb|{align}| or \verb|{IEEEeqnarray}| instead. The \verb|{eqnarray}|
environment leaves unsightly spaces around relation symbols.

Please note that the \verb|{subequations}| environment in {\LaTeX}
will increment the main equation counter even when there are no
equation numbers displayed. If you forget that, you might write an
article in which the equation numbers skip from (17) to (20), causing
the copy editors to wonder if you've discovered a new method of
counting.

{\BibTeX} does not work by magic. It doesn't get the bibliographic
data from thin air but from .bib files. If you use {\BibTeX} to produce a
bibliography you must send the .bib files. 

{\LaTeX} can't read your mind. If you assign the same label to a
subsubsection and a table, you might find that Table I has been cross
referenced as Table IV-B3. 

{\LaTeX} does not have precognitive abilities. If you put a
\verb|\label| command before the command that updates the counter it's
supposed to be using, the label will pick up the last counter to be
cross referenced instead. In particular, a \verb|\label| command
should not go before the caption of a figure or a table.

Do not use \verb|\nonumber| inside the \verb|{array}| environment. It
will not stop equation numbers inside \verb|{array}| (there won't be
any anyway) and it might stop a wanted equation number in the
surrounding equation.

\subsection{Some Common Mistakes}\label{SCM}
\begin{itemize}
\item The word ``data'' is plural, not singular.
\item The subscript for the permeability of vacuum $\mu_{0}$, and other common scientific constants, is zero with subscript formatting, not a lowercase letter ``o''.
\item In American English, commas, semicolons, periods, question and exclamation marks are located within quotation marks only when a complete thought or name is cited, such as a title or full quotation. When quotation marks are used, instead of a bold or italic typeface, to highlight a word or phrase, punctuation should appear outside of the quotation marks. A parenthetical phrase or statement at the end of a sentence is punctuated outside of the closing parenthesis (like this). (A parenthetical sentence is punctuated within the parentheses.)
\item A graph within a graph is an ``inset'', not an ``insert''. The word alternatively is preferred to the word ``alternately'' (unless you really mean something that alternates).
\item Do not use the word ``essentially'' to mean ``approximately'' or ``effectively''.
\item In your paper title, if the words ``that uses'' can accurately replace the word ``using'', capitalize the ``u''; if not, keep using lower-cased.
\item Be aware of the different meanings of the homophones ``affect'' and ``effect'', ``complement'' and ``compliment'', ``discreet'' and ``discrete'', ``principal'' and ``principle''.
\item Do not confuse ``imply'' and ``infer''.
\item The prefix ``non'' is not a word; it should be joined to the word it modifies, usually without a hyphen.
\item There is no period after the ``et'' in the Latin abbreviation ``et al.''.
\item The abbreviation ``i.e.'' means ``that is'', and the abbreviation ``e.g.'' means ``for example''.
\end{itemize}
An excellent style manual for science writers is \cite{b7}.

\subsection{Authors and Affiliations}
\textbf{The class file is designed for, but not limited to, six authors.} A 
minimum of one author is required for all conference articles. Author names 
should be listed starting from left to right and then moving down to the 
next line. This is the author sequence that will be used in future citations 
and by indexing services. Names should not be listed in columns nor group by 
affiliation. Please keep your affiliations as succinct as possible (for 
example, do not differentiate among departments of the same organization).

\subsection{Identify the Headings}
Headings, or heads, are organizational devices that guide the reader through 
your paper. There are two types: component heads and text heads.

Component heads identify the different components of your paper and are not 
topically subordinate to each other. Examples include Acknowledgments and 
References and, for these, the correct style to use is ``Heading 5''. Use 
``figure caption'' for your Figure captions, and ``table head'' for your 
table title. Run-in heads, such as ``Abstract'', will require you to apply a 
style (in this case, italic) in addition to the style provided by the drop 
down menu to differentiate the head from the text.

Text heads organize the topics on a relational, hierarchical basis. For 
example, the paper title is the primary text head because all subsequent 
material relates and elaborates on this one topic. If there are two or more 
sub-topics, the next level head (uppercase Roman numerals) should be used 
and, conversely, if there are not at least two sub-topics, then no subheads 
should be introduced.

\subsection{Figures and Tables}
\paragraph{Positioning Figures and Tables} Place figures and tables at the top and 
bottom of columns. Avoid placing them in the middle of columns. Large 
figures and tables may span across both columns. Figure captions should be 
below the figures; table heads should appear above the tables. Insert 
figures and tables after they are cited in the text. Use the abbreviation 
``Fig.~\ref{fig}'', even at the beginning of a sentence.

\begin{table}[htbp]
\caption{Table Type Styles}
\begin{center}
\begin{tabular}{|c|c|c|c|}
\hline
\textbf{Table}&\multicolumn{3}{|c|}{\textbf{Table Column Head}} \\
\cline{2-4} 
\textbf{Head} & \textbf{\textit{Table column subhead}}& \textbf{\textit{Subhead}}& \textbf{\textit{Subhead}} \\
\hline
copy& More table copy$^{\mathrm{a}}$& &  \\
\hline
\multicolumn{4}{l}{$^{\mathrm{a}}$Sample of a Table footnote.}
\end{tabular}
\label{tab1}
\end{center}
\end{table}

\begin{figure}[htbp]
\centerline{\includegraphics{fig1.png}}
\caption{Example of a figure caption.}
\label{fig}
\end{figure}

Figure Labels: Use 8 point Times New Roman for Figure labels. Use words 
rather than symbols or abbreviations when writing Figure axis labels to 
avoid confusing the reader. As an example, write the quantity 
``Magnetization'', or ``Magnetization, M'', not just ``M''. If including 
units in the label, present them within parentheses. Do not label axes only 
with units. In the example, write ``Magnetization (A/m)'' or ``Magnetization 
\{A[m(1)]\}'', not just ``A/m''. Do not label axes with a ratio of 
quantities and units. For example, write ``Temperature (K)'', not 
``Temperature/K''.

\section*{Acknowledgment}

The preferred spelling of the word ``acknowledgment'' in America is without 
an ``e'' after the ``g''. Avoid the stilted expression ``one of us (R. B. 
G.) thanks $\ldots$''. Instead, try ``R. B. G. thanks$\ldots$''. Put sponsor 
acknowledgments in the unnumbered footnote on the first page.

\section*{References}

Please number citations consecutively within brackets \cite{b1}. The 
sentence punctuation follows the bracket \cite{b2}. Refer simply to the reference 
number, as in \cite{b3}---do not use ``Ref. \cite{b3}'' or ``reference \cite{b3}'' except at 
the beginning of a sentence: ``Reference \cite{b3} was the first $\ldots$''

Number footnotes separately in superscripts. Place the actual footnote at 
the bottom of the column in which it was cited. Do not put footnotes in the 
abstract or reference list. Use letters for table footnotes.

Unless there are six authors or more give all authors' names; do not use 
``et al.''. Papers that have not been published, even if they have been 
submitted for publication, should be cited as ``unpublished'' \cite{b4}. Papers 
that have been accepted for publication should be cited as ``in press'' \cite{b5}. 
Capitalize only the first word in a paper title, except for proper nouns and 
element symbols.

For papers published in translation journals, please give the English 
citation first, followed by the original foreign-language citation \cite{b6}.

\begin{thebibliography}{00}
\bibitem{b1} G. Eason, B. Noble, and I. N. Sneddon, ``On certain integrals of Lipschitz-Hankel type involving products of Bessel functions,'' Phil. Trans. Roy. Soc. London, vol. A247, pp. 529--551, April 1955.
\bibitem{b2} J. Clerk Maxwell, A Treatise on Electricity and Magnetism, 3rd ed., vol. 2. Oxford: Clarendon, 1892, pp.68--73.
\bibitem{b3} I. S. Jacobs and C. P. Bean, ``Fine particles, thin films and exchange anisotropy,'' in Magnetism, vol. III, G. T. Rado and H. Suhl, Eds. New York: Academic, 1963, pp. 271--350.
\bibitem{b4} K. Elissa, ``Title of paper if known,'' unpublished.
\bibitem{b5} R. Nicole, ``Title of paper with only first word capitalized,'' J. Name Stand. Abbrev., in press.
\bibitem{b6} Y. Yorozu, M. Hirano, K. Oka, and Y. Tagawa, ``Electron spectroscopy studies on magneto-optical media and plastic substrate interface,'' IEEE Transl. J. Magn. Japan, vol. 2, pp. 740--741, August 1987 [Digests 9th Annual Conf. Magnetics Japan, p. 301, 1982].
\bibitem{b7} M. Young, The Technical Writer's Handbook. Mill Valley, CA: University Science, 1989.
\end{thebibliography}
\vspace{12pt}
\color{red}
IEEE conference templates contain guidance text for composing and formatting conference papers. Please ensure that all template text is removed from your conference paper prior to submission to the conference. Failure to remove the template text from your paper may result in your paper not being published.

\end{document}
